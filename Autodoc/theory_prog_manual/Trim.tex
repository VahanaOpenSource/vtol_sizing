\section{Trim Analysis}
This section details the computation of ``trim'' (steady flight) configurations, in which the vehicle acceleration components along body axes are zero and the rotor response is periodic and briefly describes the pseudo-code used to include the free-vortex wake during trim.

The ODEs of interest are strongly coupled to each other,  especially for the rotor dynamics. Explicit expressions for the accelerations (second time derivatives of displacements) as a function of forcing and velocities are lengthy and cumbersome to manipulate. One way to simplify the beam equations is to make small-angle assumptions and use an ordering scheme (Ref. \cite{Datta}), which then restricts the validity of the analysis to small angles. Alternately, it is possible to use the original form of the governing ODEs 
\begin{equation}
\label{eqn:allodes}
\vector{f}(\dot{\vector{y}}, \textrm{ } \vector{y},\textrm{ } \vector{u},\textrm{ } t\textrm{ } ) \quad = \quad \overline{\boldsymbol\epsilon} \quad = \quad \vector{0}
\end{equation}
with a \emph{class} of techniques that, given an initial guess $\vector{y}_0(t)$, obtain a solution $\vector{y}(t)$ such that $e(\overline{\boldsymbol\epsilon}) < \delta$, where $e(\overline{\boldsymbol\epsilon})$ is an error metric and $\delta$ is a user-specified threshold that is used to terminate the solution process to required numerical precision. Thus, the task of simulating vehicle dynamics is simplified to that of programming the logic for computing \emph{numerical} values of $\overline{\boldsymbol\epsilon}$ for a given $\vector{y}, \dot{\vector{y}}, \vector{u}$ and leveraging open-source subprograms from NETLIB for obtaining trim solutions and simulating maneuvering flight (Ref. \cite{CeliSoln}).

\subsection{\textbf{Definition of Trim}}
The term ``trim'' is used to refer to a steady flight condition in which the translational and angular acceleration components along and about the body axes are zero. Therefore, trim includes steady level flight, steady climbing flight, steady level turns and steady climbing/descending turns of constant radii. The concept of rotorcraft trim evolved from the corresponding definition for fixed-wing platforms, and so it is useful to define aircraft trim first.

\subsubsection{Aircraft Trim}
Trim for a fixed-wind aircraft is defined as a steady flight condition in which the control settings, orientations and velocity of the vehicle produce forces (inertial and aerodynamic) that exactly cancel out contibutions from gravity and buoyancy, thus allowing the aircraft to remain in its state of rest or uniform motion `` indefinitely ". The force distributions on a fixed-wing aircraft in trim are steady, hence the aerodynamic and inertial loads at any two instants in time will be near-identical. There may still be fluctuations in these loads at extremely high frequencies (determined by the RPMs of the various rotors inside the engine), but the amplitudes of these fluctuations are so small that their effect on aircraft trim is negligible.\\
Unlike a fixed wing, the aerodynamic and inertial loads generated by each rotor blade are not steady. In forward flight, rotor blades experience time-varying dynamic pressures and operating angles of attack, and therefore undergoes unsteady motion in response to these time-varying force and moment distributions along the span. These unsteady motions result in time-varying inertial blade loads in addition to the fluctuating aerodynamic loads, hence the forces transmitted to the airframe are vibratory in nature. With these considerations, rotor trim can be defined. 

\subsubsection{Rotor trim}
When the controls for a rotor (collective and cyclic pitch inputs) are held constant, the rotor is said to be trimmed if the blade response is periodic, i.e. it has reached steady-state, and the forces and moments, when averaged over this period, do not change over successive cycles. Often, the time period is assumed to be that reqiured for one rotor revolution, due to the cyclic variation of the free-stream velocities as seen by the blade and the kinematics of the pitch control system.

\subsubsection{Rotorcraft Trim}
Just as fixed-wing trim is not significantly affect by engine vibrations, it is assumed that, for the purposes of enforcing body force and moment equilibrium, rotorcraft trim is insensitive to the \emph{oscillatory} forces and moments transmitted to the hub. Instead, the \emph{time-averaged} forces and moments will be used to represent the contributions from rotor loads to Eqs. (\ref{eqn:bodyF1}) - (\ref{eqn:bodyM3}). This assumption is justified since the vibratory loads manifest at sufficiently large frequencies that the airframe response is negligible and the vehicle trim state is unaffected (Ref. \cite{Kim}). When the blade motion is periodic \emph{and} the time-averaged forces and moments generated by the rotor are sufficient for establishing vehicle force and moment equilibrium, the system is said to be in \emph{coupled trim} or \emph{propulsive trim}. 
 
The most general case of trim considered is a steady coordinated helical climbing turn of constant radius (Ref. \cite{CeliTurn}). This flight condition is defined by three parameters : the flight speed $V$, the flight path angle $\gamma$ (positive for climb) and the turn rate $\dot{\psi}$ (positive for nose-right turns). Using this definition,
\begin{itemize}
\item Steady level turning flight is a special case in which $\gamma$ = 0 (constant altitude)
\item Steady climbing flight is a special case in which $\dot{\psi}$ = 0
\item Steady level forward flight is a special case in which $\gamma$ = 0 and $\dot{\psi}$ = 0 
\item Hover is a special case in which $\gamma$ = 0, $V$ = 0 and $\dot{\psi}$ = 0
\end{itemize}
Mathematically, trim is enforced by imposing additional conditions on the govering ODE set Eq. (\ref{eqn:allodes}). For the rotorcraft trim problem, the differential equations reduce to nonlinear algebraic equations that may be represented as 
\begin{equation}
\label{eqn:allAE}
\vector{F}(\vector{X}) \quad = \quad \overline{\boldsymbol\epsilon}_\textrm{trim} \quad = \quad \vector{0}
\end{equation}
The problem of trim is then converted to solving a set of algebraic equations for the so-called \emph{trim unknowns}. The trim unknowns include the rotor response, vehicle attitudes, rotor induced inflow ratios and the pilot controls. Solution of the trim equations is achieved by manipulation of the trim variables $\vector{X}$ using a numerical solver (Ref. \cite{HYBRD}) until an error metric e($\rvert\overline{\boldsymbol\epsilon}\rvert_\textrm{trim}$) falls below a user-specified threshold $\delta_\textrm{trim}$. To avoid formulating an over-determined or under-determined system of equations, the number of trim variables $\vector{X}$ must be equal to the number of trim equations $\vector{F}$. The trim equations and corresponding trim variables are given in the following section. The trim solver is called \textbf{hybrd}, a NETLIB subroutine from the package \textbf{MINPACK}. The wrappers used for various types of trim are
\begin{itemize}
\item \textbf{AEResiduals} : simultaneous vehicle-rotor trim with harmonic balance
\item \textbf{AEResFET} : vehicle trim with numerical FET
\item \textbf{AEResRotorResponse} : obtain rotor response with fixed controls
\item \textbf{AEResQddot} : obtain blade accelerations for fixed controls and response
\item \textbf{AEResAirframe} : obtain airframe vibratory response 
\end{itemize}

The programming logic for all these cases is common, i.e. the calculation of the ODE residuals is unchanged. It is only the manipulation of these residuals that is unique to each of these adapters that \textbf{numerically convert the governing ODEs to algebraic (trim) equations}. 
\subsection{\textbf{Trim Equations and Trim Variables}}
\begin{itemize}
\item \textbf{The components of time-averaged fuselage translational and rotational accelerations along and about the body axes must be zero}, as given in Eqs. (\ref{eqn:bodyF1}) - (\ref{eqn:bodyM3}). Using `` T '' to represent time period for one rotor revolution, the first six trim equations are
\begin{eqnarray}
\label{eqn:vehicleeqm1}
\int_0^{T} \dot{u}_{_\textrm{F }} dt \quad = \quad \epsilon_{_\textrm{RB1}} \quad = \quad 0 \\
\int_0^{T} \dot{v}_{_\textrm{F }} dt \quad = \quad \epsilon_{_\textrm{RB2}} \quad = \quad 0 \\
\int_0^{T} \dot{w}_{_\textrm{F }} dt \quad = \quad \epsilon_{_\textrm{RB3}} \quad = \quad 0 \\
\int_0^{T} \dot{p}_{_\textrm{F }} dt \quad = \quad \epsilon_{_\textrm{RB4}} \quad = \quad 0 \\
\int_0^{T} \dot{q}_{_\textrm{F }} dt \quad = \quad \epsilon_{_\textrm{RB5}} \quad = \quad 0 \\
\label{eqn:vehicleeqm2}
\int_0^{T} \dot{r}_{_\textrm{F }} dt \quad = \quad \epsilon_{_\textrm{RB6}} \quad = \quad 0 
\end{eqnarray}
Equations (\ref{eqn:vehicleeqm1}) - (\ref{eqn:vehicleeqm2}) constitute the six trim conditions that enforce vehicle force and moment equilibrium under steady flight conditions. The corresponding trim variables are the pilot controls ($\delta_0$, $\delta_\textrm{lat}$, $\delta_\textrm{lon}$, $\delta_\textrm{ped}$) and the fuselage pitch and roll attitudes ($\phi_{_\textrm{F}}$, $\theta_{_\textrm{F}}$). The last two trim variables are indirect controls, in the sense that they cannot be immediately adjusted by the pilot. Instead, the vehicle has to be \emph{flown into} these orientations using the four direct controls that influence the lift distributions over the rotor disks. The trim equation residuals for the airframe are set in \textbf{update\_airframe\_AEResiduals}.

\item \textbf{The rotor must be trimmed}, i.e. the motions of all blades must be individually periodic. Since we assume that all blades are identical, it follows that \textbf{all blades must exhibit identical motions} with phase offsets corresponding to their relative azimuthal spacing. Therefore, the problem of obtaining the motion of all blades of a particular rotor is simplified to that of obtaining the motion of a reference blade. Without loss of generality, the first blade is chosen to be the reference blade. 

A further assumption is made at this stage to simplify the analysis - that the resulting periodic blade motion is well-represented using a Fourier series in integer multiples of the rotor frequency $\Omega$. This method is often called \emph{harmonic balance}, and can capture the dominant blade motions (with regard to flight dynamics) using the first few harmonics. \textbf{A Galerkin method with harmonic balance is used to obtain the time-resolution of the rotating blade modes}. The generalized coordinates of a blade at azimuth $\psi$ can be approximated to
\begin{equation}
\label{eqn:qharm}
\grkvec{\eta}_{j}(\psi) \quad \approx \quad \grkvec{\eta}_0 \quad + \quad \sum_{k=1}^\textrm{Nh} \left(\grkvec{\eta}_\textrm{kc} \textrm{ } \cos k\psi \quad + \quad \grkvec{\eta}_\textrm{ks} \textrm{ } \sin k\psi \right)
\end{equation}
$\grkvec{\eta}_0$ represents the steady part of the generalized coordinates, and the amplitudes of the sine and cosine components for the \mbox{`` $k^\textrm{th}$ ''} harmonic are \mbox{($\grkvec{\eta}_{kc}$ , $\grkvec{\eta}_{ks}$)}. 
The N$_\textrm{m}$(1+2 N$_\textrm{h}$) Fourier coefficients are the trim variables that define the rotor blade motions with respect to the undeformed rotating preconed axes. These Fourier coefficients are used to compute the blade deflections which are substituted into the beam equations, to yield the mode-weighted ODE residuals 
\begin{equation}
\overline{\boldsymbol\epsilon}_\textrm{blade 1} \quad = \quad \vector{f}_\textrm{beam} (\vector{y}_1, \textrm{ }\dot{\vector{y}}_1, \textrm{ } \vector{u}, \textrm{ }t) 
\end{equation}
Here, $\vector{f}_\textrm{beam}$ represents the ODEs governing rotating beam dynamics, i.e. the mode-weighted flap, lag and torsion equations. $\vector{y}_1$ represents a subset of the state vector that contains the 12 rigid-body fuselage states and the generalized coordinates (together with their first time derivatives) for the reference blade. Since we are using Galerkin's method, the corresponding trim equations are obtained by weighting the beam equations with the azimuthal shape functions and integrating over one revolution. The algebraic equation residuals corresponding to the steady, cosine and sine components of blade motions are 
\begin{align}
\grkvec{\epsilon}_\textrm{steady} \quad = \quad &\int_{0}^T \grkvec{\epsilon}_\textrm{blade 1} (t) \qquad \qquad \quad \textrm{ } dt\\
\grkvec{\epsilon}_\textrm{cos,k} \quad = \quad &\int_{0}^T \grkvec{\epsilon}_\textrm{blade 1} (t) \quad \cos k \Omega t \quad dt\\
\grkvec{\epsilon}_\textrm{sin,k} \quad = \quad &\int_{0}^T \grkvec{\epsilon}_\textrm{blade 1} (t) \quad \sin k \Omega t \quad dt
\end{align}
These residuals are computed in \textbf{update\_rotor\_AEResiduals}. 
\item \textbf{The components of helicopter linear, angular velocities along fuselage body axes, and roll and pitch attitudes must be time-invariant} \\
For trimmed flight, the vehicle must move at constant speed V. The orientation of the free-stream velocity vector relative to the airframe can be described using the spherical angles $\alpha_{_\textrm{F}}$ and $\beta_{_\textrm{F}}$ as defined in Eq. (\ref{eqn:fusab}). The translation velocity components along helicopter body axes are 
\begin{equation}
\label{eqn:uvwf}
\left.
\begin{aligned}
u_{_\textrm{F}} \quad = \quad &\textrm{V} \textrm{ }\cos \alpha_{_\textrm{F}} \cos \beta_{_\textrm{F}} \\
v_{_\textrm{F}} \quad = \quad &\textrm{V} \textrm{ }\sin \beta_{_\textrm{F}} \\
w_{_\textrm{F}} \quad = \quad &\textrm{V} \textrm{ }\sin \alpha_{_\textrm{F}} \cos \beta_{_\textrm{F}} \qquad \qquad
\end{aligned}
\right\}
\end{equation}
The rigid-body trim variables are converted to states in \textbf{interpret\_trim\_variables}, and the rotor trim variables are converted to states in \textbf{interpret\_rotor\_trimvars}. 

The helicopter yaw rate $\dot{\psi}_{_\textrm{F}}$ must be constant and the Euler pitch and roll attitudes must be time-invariant. Applying these conditions to Eqs. (\ref{eqn:pqrf}), the angular velocity components along body axes are obtained as  
\begin{equation}
\label{eqn:angturn}
\left.
\begin{aligned}
p_{_\textrm{F}} \quad = & -\dot{\psi}_{_\textrm{F}} \sin \theta_{_\textrm{F}} \qquad \\
q_{_\textrm{F}} \quad = &\quad \dot{\psi}_{_\textrm{F}} \cos \theta_{_\textrm{F}} \sin \phi_{_\textrm{F}} \qquad  \\
r_{_\textrm{F}} \quad = &\quad \dot{\psi}_{_\textrm{F}} \cos \theta_{_\textrm{F}} \cos \phi_{_\textrm{F}} \quad \qquad \quad 
\end{aligned}
\right\}
\end{equation}
At low forward speeds, the reduced dynamic pressure on the vertical stabilizer renders it ineffective for producing anti-torque. Therefore, \textbf{below a certain threshold airspeed, the helicopter is constrained to fly with zero sideslip angle}, i.e. 
\[ \beta_{_\textrm{F}} \quad = \quad 0 \]
\textbf{Above the threshold airspeed, all turns must be coordinated} to increase ride comfort and reduce the danger of entering a spin. Mathematically, turn coordination is enforced by setting the cumulative component of inertial and gravitational forces along the $\ihat{j}_{_\textrm{B}}$ direction to zero. Substituting Eqs. (\ref{eqn:uvwf}) and Eqs. (\ref{eqn:angturn}) in Eq. (\ref{eqn:bodyF2}) yields the residual of the turn coordination equation as 
\begin{equation}
\label{eqn:turncoord}
\epsilon_\textrm{coord} \quad = \quad \textrm{V} \dot{\psi}_{_\textrm{F}} \cos \beta_{_\textrm{F}} (\cos \alpha_{_\textrm{F}} \cos \theta_{_\textrm{F}} \cos \phi_{_\textrm{F}} \textrm{ }+\textrm{ } \sin \theta_{_\textrm{F}} \sin \alpha_{_\textrm{F}}) - g \sin \phi_{_\textrm{F}} \cos \theta_{_\textrm{F}}
\end{equation}
Another kinematic relationship exists between the climb angle $\gamma$, the Euler angles ($\psi$, $\theta$, $\phi$)$_{_\textrm{F}}$  and the wind angles ($\alpha$, $\beta$)$_{_\textrm{F}}$. To determine this relationship, consider the velocity components of the helicopter along fuselage body axes, as given in Eqs. (\ref{eqn:uvwf}). The velocity components along the earth-fixed axes can be obtained using the rotation matrix from body axes to gravity axes as 
\begin{equation}
\begin{Bmatrix} \dot{\textrm{x}}_{_\textrm{F}} \\\dot{\textrm{y}}_{_\textrm{F}} \\ \dot{\textrm{z}}_{_\textrm{F}} \end{Bmatrix} \quad = \quad \tee_{GB} \begin{Bmatrix} u_{_\textrm{F}} \\ v_{_\textrm{F}} \\ w_{_\textrm{F}} \end{Bmatrix}
\end{equation}
The component along $\ihat{k}_{_\textrm{G}}$ is given by the third row of the right hand side. By definition, the same velocity component is equal to 
\[ \dot{\textrm{z}}_{_\textrm{F}} \quad = \quad -V \sin \gamma\]
The negative sign accounts for the fact that $\ihat{k}_{_\textrm{G}}$ points downward and a positive $\gamma$ indicates a steady increase in altitude. \textbf{The equation of flight path} can be obtained by comparing the two expressions for $\dot{\textrm{z}}_{_\textrm{F}}$ above, and dividing by the velocity magnitude V. The residual of this trim equation is 
\begin{equation}
\label{eqn:fp}
\left.
\begin{aligned}
\epsilon_{_\textrm{FP}} \quad = \quad & \cos \alpha_{_\textrm{F}} \cos \beta_{_\textrm{F}} \sin \theta_{_\textrm{F}} \quad - \quad \sin \gamma_{_\textrm{F}} \\
- & \cos \theta_{_\textrm{F}}(\sin \beta_{_\textrm{F}} \sin \phi_{_\textrm{F}} \textrm{ + } \sin \alpha_{_\textrm{F}} \cos \beta_{_\textrm{F}} \cos \phi_{_\textrm{F}}) \qquad \qquad
\end{aligned}
\right\}
\end{equation}
The trim variables corresponding to the turn coordination and flight path equations are ($\alpha_{_\textrm{F}}$, $\beta_{_\textrm{F}}$). Perfect hover with identically zero forward speed is simulated by replacing the flight path equation with 
\begin{equation*}
\epsilon_{_\textrm{FP}} \quad = \quad \alpha_{_\textrm{F}} \qquad \qquad \textbf{\textrm{at hover}}
\end{equation*}
The kinematic consistency equations are computed in the routine \textbf{FPTCRes}. 
\item \textbf{The inflow ratios are time-invariant} when averaged over one revolution of the main rotor. The corresponding trim equation residuals are  
\begin{align*}
\epsilon_{\lambda_{TR}} \quad = \quad &\int_0^T \dot{\lambda}_{_\textrm{TR}} \textrm{ }dt \\
\epsilon_{\lambda_{0,MR}} \quad = \quad &\int_0^T \dot{\lambda}_{0_\textrm{MR}} \textrm{ }dt \\
\epsilon_{\lambda_{1c,MR}} \quad = \quad &\int_0^T \dot{\lambda}_{\textrm{1c}_\textrm{MR}} \textrm{ }dt \\
\epsilon_{\lambda_{1s,MR}} \quad = \quad &\int_0^T \dot{\lambda}_{\textrm{1s}_\textrm{MR}} \textrm{ }dt 
\end{align*}
\end{itemize}
The ODEs governing the inflow dynamics are converted to trim equations in the routine \textbf{update\_inflow\_AEResiduals}.
\subsection{\textbf{Free-Vortex Wake Model in Trim}}
When the free wake model is used in trim, all the trim conditions given in the previous sections are enforced. The main rotor inflow equations are initially used to generate a starting guess for the trim controls, rotor response and fuselage orientations. Once trim is achieved with dynamic inflow, the main rotor inflow equations are removed from the trim equations and a `` loose-coupling '' procedure is used to periodically exchange information over one rotor revolution between the aerodynamics and rotor/flight dynamics. Reference \cite{Alfred} provides details on the loose-coupling trim procedure, and a brief summary is given here for completeness.

\begin{enumerate}
\item With the trim controls, fuselage velocity and blade motions from the previous iteration, the free wake solution is marched forward in time until the L1 norm of the inflow over the rotor disk reduces below a threshold value $\delta_\textrm{inflow}$. This operation is performed by the routine \textbf{converge\_wake\_inflow}.
\item The inflow distribution over the rotor disk is computed from the converged free wake geometry and frozen. This step is performed by the routine \textbf{rindvt} (wake folder). 
\item Using this \textit{frozen} inflow distribution, the trim procedure is applied a solution for simultaneous vehicle equilibrium and rotor response periodicity. Once trim is achieved with the inflow distribution from step 2, the structural/flight dynamics are \emph{frozen}. This procedure is performed by the routine \textbf{AESolver}.
\item Steps 1-3 are repeatedly performed until the L1 norm of trim variables (excluding 2/rev and higher rotor harmonics) reduce below a threshold $\delta_{_\textrm{TV}}$. This operation is controlled by the routine \textbf{iterative\_trim}.
\end{enumerate}

\subsection{\textbf{Galerkin vs. Rayleigh-Ritz, FET vs. Harmonic Balance}}
There are two key differences between \textbf{UMARC} and \textbf{HeliUM}. 
\begin{itemize}
\item \textbf{The first difference is the time resolution of the rotor modes}. While HeliUM uses harmonic balance to obtain simultaneous rotor-vehicle trim, \textbf{UMARC} uses a Finite Element in Time method, using local polynomials to express the time-variation of blade motion. This choice of time shape function (trigonometric vs. Lagrangian interpolation polynomials) implies that blade inertial loads are computed differently in the two codes, and each method exhibits its own strengths and deficiencies. For example, Lagrangian time shape functions yield necessarily discontinuous accelerations across time nodes, while harmonic balance \textit{forces} continuity in displacement, velocity and acceleration. The validity of harmonic balance comes into question when there is impulsive loading on the blade at moderate advance ratios due to dynamic stall or advancing blade compressibility.
\item \textbf{The second key difference is the technique used to manipulate the rotor dynamics and recast the equations into a solvable form.} The choice of a semi-implicit (partially numerical) formulation of the beam equations in \textbf{HeliUM} allows for using a state-space representation throughout the analysis, which eliminates duplication in programming. However, the dynamics are ``hidden'' in the numbers, which are generated only during run-time. For beginners, the subtleties in the trim process, i.e. a numerical reduction of the governing ODEs to algebraic equations is not immediately obvious. The biggest advantage of this approach is the sheer power and flexibility afforded to the developer in terms of adding features in a modular fashion. In \textbf{UMARC}, the analytically derived rotating beam equations of motion are recast into a linearized form with select Taylor-series expansions about zero deflection. This formulation is easier to grasp, since terms like acceleration and Coriolis force appear explicitly in the equations of motion. This ``explicitness'' is numerically more efficient to implement, since it replaces sines and cosines with polynomial approximations. However, adding additional physical effects (e.g. time-varying RPM fluctuations, hub angular velocities/accelerations, free-stream flow velocity acceleration) requires re-derivation of parts of the analysis, and the lead time for expanding program capabilities is considerable. The art of cherry-picking cross-couplings in flap, lag and torsion based on an ordering scheme quickly balloons into an exercise in superhuman book-keeping, and is extremely error-prone. Further, programming these long expressions (the physical origins of which are lost in the reduction of sines and cosines to polynomials) does not leave a clean trail of code to follow. 
\end{itemize}

In this section, I describe how to combine the best of both worlds - i.e. preserve the modularity of a state-space representation with large deflections, while simultaneously obtaining accurate inertial loads using Finite Element in Time, using a modification of the \textbf{UMARC} trim process.

\subsection{\textbf{Conventional trim process for rotors}}
\textbf{Disclaimer}: I have skipped the non-essential steps in an attempt to be concise. Wherever possible, higher spatial and time derivatives must be eliminated using integration by parts. Using a simple example of a rotor blade undergoing elastic flap deflections, the \textbf{UMARC} rotor trim process is explained here. The rotor blade is modeled as a rotating Euler-Bernoulli beam, with its flap dynamics governed by the PDE
\begin{equation}
\left( EI w''\right)'' \spc + \spc m\ddot{w} \spc - \spc \left(T w'\right)' \quad = \quad f_z(w, \dot{w}, \theta_\textrm{con}, t)
\end{equation} 
The blade deflections are split into a space-dependent and time-dependent part, assuming separation of variables of the form 
\begin{equation}
w(x,t) \quad = \quad \sum q_j(t) \spc V_i(x)
\end{equation}
The aim of rotor trim is to find the rotor motions $w(x,t)$ such that the governing PDE is satisfied \textit{not at every point on the beam}, but instead in an \textit{approximate} sense (see Sections \ref{sec:galerkin} and \ref{sec:modes}). (While \textbf{UMARC} uses a Rayleigh-Ritz forumation to obtain the blade governing equations, a Galerkin method yields identical results for the same trial functions) Thus, the aim is to satisfy the following \textit{ODEs} 
\begin{equation}
\int_0^L \left[\spc \left( EI w''\right)'' \spc + \spc m\ddot{w} \spc - \spc \left(T w'\right)' \spc \right] \phi_i(x) dx \quad = \quad \int_0^L f_z \spc \phi_i(x) dx 
\end{equation}
There are as many ODEs as there are weighting functions $\phi_i$. The right hand side of the \textit{weak} formulation is the external loading (aerodynamics). Since the aerodynamic loads depend on blade motion velocities, there are ``hidden'' dependencies of $f_z$ on $w, \dot{w}$ and $\ddot{w}$. To solve for $w$ ``analytically'', these dependencies are ``exposed'' using analytical Taylor-series expansions for $f_z$ as 
\begin{equation}
f_z \quad = \quad f_0 \spc + \spc + \frac{df}{dw} (w - w_0) \spc + \spc \frac{df}{d\dot{w}} (\dot{w} - \dot{w}_0) \spc + \spc \cdots
\end{equation}
Applying separation of variables, the weak formulation can be written as a spring-mass-damper type second-order ODE, given by 
\begin{equation}
\vector{M} \ddot{\vector{q}} \spc + \spc \vector{C} \dot{\vector{q}} \spc + \spc \vector{K} \vector{q} \quad = \quad \vector{F}_0(t) \spc + \spc \vector{F}_\textrm{non-lin}(t)
\end{equation}
The terms \textbf{F}$_0(t)$ is dependent only on control inputs, and \textbf{F}$_\textrm{non-lin}(t)$ contains the ``higher-order terms'' resulting from the Taylor-series expansion. The mass and stiffness matrices depend only on blade properties, and can be used to perform modal reduction (Section \ref{sec:modes}) to yield the modal equations
\begin{equation}
\left[\overline{\vector{M}}\right] \ddot{\grkvec\eta} \spc + \spc \left[\overline{\vector{C}}\right] \dot{\grkvec\eta} \spc + \spc \left[\overline{\vector{K}}\right] \grkvec\eta \quad = \quad \overline{\vector{F}}(t)
\end{equation}
Here, $\left[\overline{\spc}\right]$ represents a modal matrix, obtained from the \textit{nodal} matrices using the transformation
\begin{equation}
\left[\vector{modal}\right] \quad = \quad \vector{V}^\textrm{T} \spc \left[\vector{nodal}\right] \spc \vector{V}
\end{equation}
Here, $\vector{V}$ represents a matrix of eigenvectors.

Using these finite numbers of ODEs, an approximate solution is applied again to obtain the time resolution of the rotor modes $\grkvec\eta$. Instead of solving the modal equations for \textit{each} point in time, the time history of rotor modes over one revolution (in rotor trim) is expanded as a linear combination of trial functions as 
\begin{equation}
\grkvec\eta(t) \quad = \quad \sum H_j(t) \grkvec\xi_j
\end{equation}
The approximate solution is obtained by satisfying the following condition
\begin{equation}
\left[ \int_0^T \vector{H}^\textrm{T} \left( \left[\overline{\vector{M}}\right] \ddot{\vector{H}} \spc + \spc \left[\overline{\vector{C}}\right] \dot{\vector{H}} \spc + \spc \left[\overline{\vector{K}}\right] \vector{H} \right) \spc dt \right] \grkvec\xi \quad = \quad \int_0^T \vector{H}^\textrm{T} \spc \overline{\vector{F}} \spc dt
\end{equation}
The coefficient matrix on the left hand side that multiplies $\grkvec\xi$ is evaluated numerically from the modal mass, damping and stiffness matrices, since the non-linearities are packaged into locally-linearized additions to the left hand side. The right hand side is also evaluated numerically from the control inputs and rotor operating condition, and a system of linear equations is solved to obtain the time shape function coefficients of the rotor modes. Since the blade accelerations are discontinuous when using finite elements in time with Lagrangian or Hermitian shape functions, they are instead obtained by inverting the modal equations, i.e.
\begin{equation}
\ddot{\grkvec\eta} \quad = \quad \left[\overline{\vector{M}}\right]^{-1} \left( \spc \overline{\vector{F}}(t) \spc - \spc \left[\overline{\vector{C}}\right] \dot{\grkvec\eta} \spc - \spc \left[\overline{\vector{K}}\right] \grkvec\eta \spc \right)
\end{equation}
To recap,
\begin{itemize} 
\item Linearization of aerodynamic, structural and inertial loads, combined with the small-angles assumption provides analytical expressions that can be manipulated to obtain the rotor response.
\item Replacing trigonometric expressions with polynomials increases computational efficiency, while preserving more than 95\% accuracy.
\item \textbf{An ordering scheme is required to deal with lengthy analytical expressions, and the formulation must be re-derived for different materials (e.g. composites)}
\item In \textbf{UMARC}, the processes of vehicle trim and rotor trim are isolated from each other. Using the blade mode shapes and the coefficients of the individual time shape functions $\grkvec\eta$, the airloads and inertial loads can be computed for a given set of controls. These controls are iteratively adjusted starting from an initial guess using a Jacobian trim process. \textbf{Thus, the rotor trim process is hand-crafted using explicit analytical expressions, but the vehicle trim process is handled numerically.}
\end{itemize}

\subsection{\textbf{Numerical FET}}
The basic idea is to apply \textbf{UMARC}'s \textit{vehicle trim} numerical approach to also obtain the rotor motions using the following steps
\begin{itemize}
\item Assume that $w(x,t) \quad = \quad \sum \sum \xi_{ij} \spc V_i(x) \spc H_j(t)$ is available as an initial guess (instead of a quantity that needs to be ``solved for''). $V_i(x)$ is the spatial trial function, $H_j(t)$ is the temporal trial function and $\xi_{ij}$  is the time shape function coefficient of a rotor blade mode. 
\item For a given set of $V_i(x)$, $H_j(t)$, find $\xi_{ij}$ such that the following set of \textit{approximate} equations are satisfied
\begin{equation}
\int_0^T H_j(t) \int_0^L \left(\textrm{LHS} - \textrm{RHS}\right) V_i(x) \spc dx \spc dt \quad = \quad  \epsilon_{ij} \quad = \quad 0
\end{equation}
\item Starting from an initial guess, iteratively adjust $\xi_{ij}$ to minimize $|\epsilon_{ij}|$, $i = 1, 2, \cdots N_m, \quad j = 1, 2, \cdots N_t$. The advantage of this approach is that the deflections are ``known'' at every time instant and point on the beam, and the need for small-angle approximations or analytical expressions are eliminated entirely. The present beam model is therefore spatially ``exact''.
\item By replacing the time shape functions $H_j(t)$ with trigonometric functions, we also preserve the ability to switch to harmonic balance without changing the trim structure.
\item The calculation of inertial loads and vehicle trim process uses the same framework, and the numerical routines need not be duplicated. 
\item The rotor dynamics are, in effect, invisible to the trim solver. This modularity allows for adding additional complexity to the simulation (e.g. axial degrees of freedom, trailing-edge flaps or multiple rotors) without having to modify the rest of the code.
\end{itemize}

\subsubsection{Accelerated FET using Harmonic Balance}
The main disadvantage of FET is that the rotor and vehicle trim processes must be decoupled due to the discontinuities in the second time derivatives of the time shape functions. Obtaining the vehicle trim Jacobian is the most computationally expensive step in the trim process, since each set of trial controls require a trimmed rotor to obtain accurate hub loads. The key idea, therefore, is to speed up the computation of this Jacobian so that faster solutions may be obtained with FET. 
By contrast, harmonic balance enforces continuity in accelerations, which allows for simultaneous vehicle and rotor trim at the cost of ignoring step changes in aerodynamic forcing (e.g. advancing blade shocks and retreating blade stall). This capability allows us to write the trim equations for the coupled rotor-body system as 
\begin{equation}
\begin{Bmatrix} \grkvec\epsilon_\textrm{NR} \\ \grkvec\epsilon_\textrm{R} \end{Bmatrix} \quad = \quad \begin{Bmatrix} \vector{f}_\textrm{NR} (\vector{x}_\textrm{NR}, \spc \vector{x}_\textrm{R}) \\ \vector{f}_\textrm{R} (\vector{x}_\textrm{NR}, \spc \vector{x}_\textrm{R}) \end{Bmatrix}
\end{equation}
$\grkvec\epsilon$ represents the residuals of the trim equations, $\vector{x}$ denotes the trim variables and the subscripts denote non-rotating (NR) and rotating (R) components, i.e. everything but the rotor motions, and the rotor motions. Using finite-differences, the Jacobian matrix \textbf{J} may be partitioned as
\begin{equation}
\begin{Bmatrix} \grkvec\epsilon_\textrm{NR} \\ \grkvec\epsilon_\textrm{R} \end{Bmatrix} \quad = \quad \begin{bmatrix} \vector{J}_{11} & \vector{J}_{12} \\ \vector{J}_{21} & \vector{J}_{22} \end{bmatrix} \spc \begin{Bmatrix} \Delta \vector{x}_\textrm{NR} \\ \Delta \vector{x}_\textrm{R}  \end{Bmatrix}
\end{equation}
The linearized rotor trim equations may be written as
\begin{equation}
\grkvec\epsilon_\textrm{R} \quad = \quad \textbf{J}_{21} \spc \Delta \textbf{x}_\textrm{NR} \spc + \spc \textbf{J}_{22} \spc \Delta \textbf{x}_\textrm{R}
\end{equation}
The linearized vehicle trim equations may be written as
\begin{equation}
\grkvec\epsilon_\textrm{NR} \quad = \quad \textbf{J}_{11} \spc \Delta \textbf{x}_\textrm{NR} \spc + \spc \textbf{J}_{12} \spc \Delta \textbf{x}_\textrm{R}
\end{equation}
For harmonic balance, the entire Jacobian \textbf{J} is used for simultaneous rotor-vehicle trim. For FET, we require a \textit{reduced} Jacobian that represents the sensitivity of integrated vehicle loads to changes in vehicle trim variables, \textit{with a trimmed rotor}. Expanding on the operative word,
\[ \textrm{\textbf{For a trimmed rotor}} \qquad \grkvec\epsilon_\textrm{R} \quad = \quad \vector{0} \]
Thus,
\[ \textbf{J}_{21} \spc \Delta \textbf{x}_\textrm{NR} \spc + \spc \textbf{J}_{22} \spc \Delta \textbf{x}_\textrm{R} \quad = \quad \vector{0} \]
The rotor trim variables may be obtained in terms of the vehicle trim variables as 
\[ \Delta\textbf{x}_\textrm{R} \quad = \quad -\textbf{J}_{22}^{-1} \spc \textbf{J}_{21} \spc \Delta\textbf{x}_\textrm{NR} \]
Substituting this expression in the linearized vehicle trim equations to yield
\begin{align*}
\grkvec\epsilon_\textrm{NR} \quad = \quad &\textbf{J}_{11} \spc \Delta \textbf{x}_\textrm{NR} \spc - \spc \textbf{J}_{12} \spc \textbf{J}_{22}^{-1} \spc \textbf{J}_{21} \spc \Delta\textbf{x}_\textrm{R} \\
\spc = \quad & \left(\spc \textbf{J}_{11} \spc - \spc \textbf{J}_{12} \spc \textbf{J}_{22}^{-1} \spc \textbf{J}_{21} \spc \right) \spc \Delta \textbf{x}_\textrm{R}
\end{align*}
An inspection of the above expression reveals that the FET trim Jacobian is 
\begin{equation}
\textbf{J}_\textrm{FET} \quad = \quad \textbf{J}_{11} \spc - \spc \textbf{J}_{12} \spc \textbf{J}_{22}^{-1} \spc \textbf{J}_{21}
\end{equation}
Partitions of the trim Jacobians from Harmonic Balance can be manipulated and reduced for use with FET, therefore bypassing the most time-consuming part of the computations. This manner of reduction of a sensitivity matrix is similar to the procedure used to obtain a stability derivative model of a helicopter. 
